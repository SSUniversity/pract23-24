\documentclass[master, och, pract]{SCWorks}
% Тип обучения (одно из значений):
%    bachelor   - бакалавриат (по умолчанию)
%    spec       - специальность
%    master     - магистратура
% Форма обучения (одно из значений):
%    och        - очное (по умолчанию)
%    zaoch      - заочное
% Тип работы (одно из значений):
%    coursework - курсовая работа (по умолчанию)
%    referat    - реферат
%    otchet     - универсальный отчет
%    nirjournal - журнал НИР
%    diploma    - дипломная работа
%    pract      - отчет о научно-исследовательской работе
%    autoref    - автореферат выпускной работы
%    assignment - задание на выпускную квалификационную работу
%    review     - отзыв руководителя
%    critique   - рецензия на выпускную работу
% Включение шрифта
%    times      - включение шрифта Times New Roman (если установлен)
%                 по умолчанию выключен
\usepackage{preamble}

\begin{document}

% Кафедра (в родительном падеже)
\chair{Информатики и программирования}

% Тема работы
\title{Разработка платформы единого резюме}

% Курс
\course{2}

% Группа
\group{273}

% Факультет (в родительном падеже) (по умолчанию "факультета КНиИТ")
% \department{факультета КНиИТ}

% Специальность/направление код - наименование
% \napravlenie{02.03.02 "--- Фундаментальная информатика и информационные технологии}
\napravlenie{02.04.03 "--- Математическое обеспечение и администрирование информационных систем}
% \napravlenie{09.03.01 "--- Информатика и вычислительная техника}
% \napravlenie{09.03.04 "--- Программная инженерия}
% \napravlenie{10.05.01 "--- Компьютерная безопасность}

% Для студентки. Для работы студента следующая команда не нужна.
% \studenttitle{Студентки}

% Фамилия, имя, отчество в родительном падеже
\author{Кулакова Максима Сергеевича}

% Руководитель НИР
\nirtitle{к.\,э.\,н., доцент} % степень, звание
\nirname{Л.\,В.\,Кабанова}

% Заведующий кафедрой
\chtitle{к.\,ф.-м.\,н.} % степень, звание
\chname{М.\,В.\,Огнева}

% Научный руководитель (для реферата преподаватель проверяющий работу)
\satitle{к.\,э.\,н., доцент} %должность, степень, звание
\saname{Л.\,В.\,Кабанова}

% Руководитель практики от организации (только для практики, для остальных типов
% работ не используется)
\patitle{к.\,э.\,н., доцент}
\paname{Л.\,В.\,Кабанова}

% Семестр (только для практики, для остальных типов работ не используется)
\term{1}

% Наименование практики (только для практики, для остальных типов работ не
% используется)
\practtype{производственная (научно-исследовательская работа)}

% Продолжительность практики (количество недель) (только для практики, для
% остальных типов работ не используется)
\duration{18}

% Даты начала и окончания практики (только для практики, для остальных типов
% работ не используется)
\practStart{01.09.2023}
\practFinish{14.01.2024}

% Год выполнения отчета
\date{2023}

\maketitle

% Включение нумерации рисунков, формул и таблиц по разделам (по умолчанию -
% нумерация сквозная) (допускается оба вида нумерации)
\secNumbering

\tableofcontents

% Раздел "Обозначения и сокращения". Может отсутствовать в работе
% \abbreviations
% \begin{description}
%     \item ... "--- ...
%     \item ... "--- ...
% \end{description}

% Раздел "Определения". Может отсутствовать в работе
% \definitions

% Раздел "Определения, обозначения и сокращения". Может отсутствовать в работе.
% Если присутствует, то заменяет собой разделы "Обозначения и сокращения" и
% "Определения"
% \defabbr

% Ссылка на источник в тексте
% \cite{}

\intro
Вопрос поиска работы всегда находился перед лицом человека, ведь работа должна 
приносить не только деньги, но и удовлетворение физических и психологических 
потребностей человека. Наиболее актуальной проблемой со стороны соискателя является 
то, где искать необходимого ему работодателя, а также с какой стороны преподнести 
свои навыки и умения, чтобы в ближайшие дни занимать рабочее место своей мечты.


Задачами работы являются следующие пункты:
\begin{enumerate}
    \item Масштабирование платформы до клиент-серверного приложения;
    \item Конфигурация основного функционала; 
    \item Реализация взаимодействия по API.

    \item Обзор научной литературы (в том числе научно-технической) по теме 
    <<Разработка платформы единого резюме>>;
    \item Рассмотрение и анализ существующих платформ для создания резюме;
    \item Формулировка собственных методов разработки единой платформы резюме;
    \item Подведение итогов проведенной научно-исследовательской работы.
\end{enumerate}


% После введения — серии \section, \subsection и т.д.
\newpage
\section{Анализ научной литературы}
Рассматриваемая литература будет затрагивать тему аспектов составления резюме, 
принципы их составления и критерии, по которым работодателю с наибольшей вероятностью 
понравится грамотно составленное резюме. После проведения анализа данной темы нам 
предоставится возможность выделить основные пункты, которые будут учитываться 
при разработке собственной единой платформе резюме.

Для начала стоит рассмотреть научные статьи, связанные с доказательством важности 
правильного составления резюме в настоящее время, и какие изменения  оно претерпевает. 
В статье К.В. Косолаповой <<Типологические особенности современного резюме на английском 
языке>> автор выделяет основные пункты в резюме, которые было принято считать достаточными:
\begin{enumerate}
    \item Полные ФИО;
    \item Возраст;
    \item Место проживания на текущий момент;
    \item Место учёбы, уровень образования;
    \item Список умений;
    \item Опыт работы (при его наличии);
    \item Контактные данные.\cite{Gridneva_2021}
\end{enumerate}



\section{Реализация платформы единого резюме}

Сделаем структуру приложения
\begin{enumerate}
    \item app
    \item component
    \item config
    \item data-template
    \item lib
    \item public
    \item styles
    \item types
    \item root
\end{enumerate}

Без авторизации нельзя перейти на страницы в middleware.ts
\begin{minted}[fontsize=\small, breaklines=true, style=bw, linenos]{ts}
export const config = { matcher: ['/profile/:path*', '/hh', '/resume','/protected/:path*'] }
\end{minted}

process.env для хранения переменных, которые нельзя публиковать и вызывать их из кода, 
например process.env.Variable


\newpage
\conclusion
В результате проведения исследовательской работы были приобретены навыки анализа 
качества и эффективности научной литературы в области разработки сервисов 
с автоматическим обновлением данных, достигнут навык анализирования конкурентных 
платформ для создания резюме и сформулирован собственный метод разработки 
единой платформы резюме, тем самым было достигнуто полное выполнение поставленных задач.

В качестве объектов анализа научной литературы выступили статьи по темам составления 
резюме, их анализа со стороны социологии, а также статьи, в которых рассматриваются 
инструменты для веб-разработки. С учётом проведённого анализа научной литературы 
были составлены основные требования для разработки будущей платформы как с технической 
стороны, так и со стороны гуманитарно-социальных наук.

Анализ конкурентных платформ позволил выявить слабые стороны существующих сервисов, 
исправление которых возможно реализовать в разработке собственной единой платформы 
резюме при условии его дальнейшего масштабирования.



% Библиографический список, составленный вручную, без использования BibTeX
%
% \begin{thebibliography}{99}
%   \bibitem{Ione} Источник 1.
%   \bibitem{Itwo} Источник 2
% \end{thebibliography}

% Отобразить все источники. Даже те, на которые нет ссылок.
% \nocite{*}

% Меняем inputencoding на лету, чтобы работать с библиографией в кодировке
% `cp1251', в то время как остальной документ находится в кодировке `utf8'
\inputencoding{cp1251}
\bibliographystyle{gost780uv}
\bibliography{thesis}
\inputencoding{utf8}

% При использовании biblatex вместо bibtex
% \printbibliography

% Окончание основного документа и начало приложений Каждая последующая секция
% документа будет являться приложением
\appendix

\end{document}
